\documentclass[prd,amsfonts,aps,nofootinbib,notitlepage,11pt,superscriptaddress]{revtex4-1}
%%% RIGHT WAY TO USE UTF8 %%%
%https://tex.stackexchange.com/a/54597/59580
\usepackage{ifxetex}
\ifxetex
  \usepackage{fontspec}
\else
  \pdfoutput=1
  \usepackage[utf8]{inputenc}
\fi
%
\usepackage{amsmath,amssymb}
\usepackage{graphicx}
\usepackage{cancel}
\usepackage[colorlinks=true]{hyperref}
\usepackage{color}
\usepackage{array,multirow} 
\usepackage{subcaption}

\begin{document}

\section{Anomaly cancelation}
From~\cite{Calle:2019mxn}:
física por la mañana
We use $f$ ($f$) to denote the general $\operatorname{U}(1)_X$ generation-independent charge assignments of the field $f_R$ ($F_L$). The three linear anomalies in $\operatorname{U}(1)_X$~\cite{Campos:2017dgc}
\begin{align}
  \label{eq:anolin}
  \left[\mathrm{SU}(3)_{C}\right]^{2} \mathrm{U}(1)_{X} :& &
[3 u+3 d]-[3 \cdot 2 q]=&0, \nonumber\\
\left[\mathrm{SU}(2)_{L}\right]^{2} \mathrm{U}(1)_{X} :&&
-[2 l+3 \cdot 2 q]=&0, \nonumber\\
\left[\mathrm{U}(1)_{Y}\right]^{2} \mathrm{U}(1)_{X} :&&
{ \left[(-2)^{2} e+3\left(\tfrac{4}{3}\right)^{2} u+3\left(-\tfrac{2}{3}\right)^{2} d\right]-\left[2(-1)^{2} l+3 \cdot 2\left(\tfrac{1}{3}\right)^{2} q\right]}=&
 0,
  \end{align}
allows to express three $X$-charges in terms of the other two
\begin{align}
  u=&-e+\frac{2l}{3}\,,& d=& e-\frac{4l}{3}\,,& q=& -\frac{l}{3}\,.
\end{align}
The quadratic anomaly condition is automatically satisfied, while the mixed gauge-gravitational and cubic anomalies depend of any extra singlet quiral fermions of zero hypercharge, like the right-handed counterpart of the Dirac neutrinos. For $N$ extra quiral fields with $X$-charge $n_{\alpha}$, these conditions read
\begin{align}
  \left[\text{Grav}\right]^{2} \mathrm{U}(1)_{X}:\ \sum_{\alpha=1}^N n_{\alpha}+3 (e-2l)=&0\,, &  \left[\mathrm{U}(1)_{X}\right]^{3}:\ \sum_{\alpha=1}^N n_{\alpha}^3+3 (e-2l)^3=&0\,. &
\end{align}
We choose the solutions with $r\equiv e-2l$, such that
\begin{align}
  \label{eq:anolam}
  \sum_{\alpha=1}^{N} n_{\alpha}=&-3 r\,,   & \sum_{\alpha=1}^{N} n_{\alpha}^3=&-3 r^3\,.
\end{align}
The full set of anomaly free SM $X$-charges in terms of two parameters~\cite{Appelquist:2002mw,Campos:2017dgc,Allanach:2018vjg} that we choose as $l$ and $r$, is just
\begin{align}
  u=&-r-\frac{4l}{3}\,,&d=&r+\frac{2l}{3}\,,&q=&-\frac{l}{3}\,,&e=&r+2l\,,&h=&-r-l\,.
  \label{Eq:SMCharges}
\end{align}
where the condition in the charged lepton Yukawa couplings have been used to fix $h$, and is automatically consistent with the conditions in the quark Yukawa couplings. By setting $l=0$ in the previous equations, we can define the Abelian symmetry in which only the right-handed charged fermions have non-vanishing $X$-charges as $\operatorname{U}(1)_R$. Then the general anomaly free two-parameter solution can be written as
\begin{align}
  X(r,l)=r R- l Y\,.
\end{align}


If we now change $f\to f'= f/r$ for all the charged fermion $X$-charges~\cite{Allanach:2018vjg}, the
first set of anomaly cancellation conditions Eq.~\eqref{eq:anolin} remains
invariant, and without lost of generality it is always possible to
normalize the solutions such that the last set Eq.~\eqref{eq:anolam} is
just
\begin{align}
  \label{eq:anolamnor}
   \sum_{\alpha=1}^{N} n_{\alpha}^{\prime}=&-3\,,   & \sum_{\alpha=1}^{N} n_{\alpha}^{\prime\, 3}=&-3\,.
\end{align}
For example, the solution with $r=3$: $n_{\alpha}=\left( -2,-2,-4,-1 \right)$~\cite{Appelquist:2002mw} can be easily normalized to the form in Eq.~\eqref{eq:anolamnor} with $f\to f/3$
to $n_{\alpha}'=\left( -2/3,-2/3,-4/3,-1/3 \right)$
as used in Ref.~\cite{Patra:2016ofq}. In this way, without lost of generality, we will work with the normalized solution in terms of a single parameter~\cite{Jenkins:1987ue,Oda:2015gna,Okada:2018tgy} that we choose to be $l$, by setting $r=1$ as summarized in column $\operatorname{U}(1)_X$ of Table~\ref{tab:partcont3}, which is just
\begin{align}
  X(l)=R-l\, Y\,.
\end{align}
In particular, this includes the solution  $n_{\alpha}=(-4,-4,+5)$~\cite{Appelquist:2002mw}. 



%
\begin{table}
  \centering
  \begin{tabular}{|c|c|c|c|c|c|c|c||c|}
    \hline  
    Fields     & $\operatorname{SU}(2)_L$ &  $\operatorname{U}(1)_Y $ & $\operatorname{U}(1)_{X}$& $\operatorname{U}(1)_{B-L}$& $\operatorname{U}(1)_R$& $\operatorname{U}(1)_D$& $\operatorname{U}(1)_G$ & $\operatorname{U}(1)_{\mathcal{D}}$\\ \hline
$L $     & $\boldsymbol{2}$ & $-1$ & $l$      &  $-1$&    $0$ &  $-3/2$&  $-1/2$ & $0$ \\    
$d_R $   & $\boldsymbol{1}$ & $-2/3$ & $1+2l/3$ &  $1/3$&    $1$&  $0$&    $2/3$ & $0$\\
$u_R $   & $\boldsymbol{1}$ & $+4/3$ & $-1-4l/3$&  $1/3$&   $-1$&  $1$&   $-1/3$ & $0$ \\
$Q $     & $\boldsymbol{2}$ & $1/3$ & $-l/3$   & $1/3$&    $0$&  $1/2$&  $1/6$ & $0$ \\
$e_R $   & $\boldsymbol{1}$ & $-2$   & $1+2l$   &  $-1$&    $1$ &  $-2$&  $0$ & $0$\\\hline
$H $     & $\boldsymbol{2}$ & $1$  & $-1-l$   &  $0$&    $-1$ &  $1/2$&  $-1/2$ & $0$ \\\hline
$N_\alpha$& $\boldsymbol{1}$ & $0$   & $n_\alpha$& $n_\alpha$&  $n_\alpha$ & $n_\alpha$& $n_\alpha$ & $n_\alpha$\\\hline
  \end{tabular}
  \caption{General one-parameter solution with some examples of rational solutions
    ($X=B-L,R,D,G$ and $\mathcal{D}$) for the radiative type-I seesaw realization
    of the effective operator $\mathcal{O}_{6D}$ for Dirac neutrino masses. The
    last column requieres the condition $\sum_{\alpha=1}^N n_\alpha=0$\,.}
    \label{tab:partcont3}
\end{table}
%


\section{Discrete symmetries}

The possible charge assignments to a field $\phi$ under a $Z_N$ symmetry are
\begin{align}
    1,w, w^2, ...,w^{N-1}, \,\,\,\text{with}\,\,\,w=\exp(i2\pi/N). 
\end{align}
We assume the existence of $k$ scalar complex fields $\phi_\alpha$, $\alpha=1,2,...,k$, each transforming as
\begin{align}
    \phi_\alpha\sim w^\alpha,  \,\,\,\text{with}\,\,\,k\leq N/2.
%    \begin{cases}
%    N/2,\,\,\,\text{for}\,\,\, N \,\,\,\text{odd},\\
%    N/2-1,\,\,\,\text{for}\,\,\, N \,\,\,\text{even}.    
%    \end{cases} 
\end{align}
Note that $(w^\alpha)^*=w^{-\alpha}=w^{N-\alpha}$. 

\subsection{Single dark sector particle}
For a Dark sector containing only a single particle, $Z_N$ can be emerged only for $N=2$ or $3$, according to the following renormalizable term in the Lagrangian.

Consider one $\operatorname{U}(1)_X$ which is broken by a singlet scalar field $S_N$ and one dark scalar field $\phi_1$ where the subscript indicates the $X$-charge of the field. In addition to the self-conjugate terms the gauge symmetry allows the following term~\cite{Batell:2010bp}
\begin{align}
  \label{eq:zncon}
  \mathcal{L}\varpropto S_N^{\dagger}\phi_1^N + \text{h.c}\,.
\end{align}

When $S_N$ acquires a vev, the Lagrangian preserves a discrete $Z_N$ subgroup of $\operatorname{U}(1)_X$ under which the dark scalar field transform as
\begin{align}
  \phi\to \phi' = &\operatorname{e}^{2\pi i k/N} \phi_1\,,& k=&0,1,2,\ldots N-1\,.
\end{align}
Then the remnant symmetry will be exactly conserved at all orders
including non-renormalizable terms.  Note that the $Z_N$ symmetry can
also appears accidentally. In this case, in contrast to the discrete
gauge symmetries, these accidental symmetries may be violated at the
non-renormalizable level, and consequently, the DM stability can be
only granted at the renormalizable level~\cite{Batell:2010bp}.

In general, the term in eq.~\eqref{eq:zncon}


For the $Z_2$ imposed usually by hand, we can identify this a gauge discrete symmetry associated to Lagrangian of the Singlet Scalar Dark Matter model (SSDM), $\phi_1$, supplemented by the interaction with the singlet scalar $S_2$ which spontaneously breaks some $\operatorname{U}(1)_X$ gauge symmetry. The relevant term in the Lagrangian is
\begin{align}
  \mathcal{L}\subset \lambda S_2^{*} \phi_1 \phi_1+ \text{h.c.}\,.
\end{align}
Note that in this case $\phi_1$  needs to be complex. However, this Lagrangian term induces a mass splitting between the real an imaginary components of $\phi_1$. In this way we can have the usual SSDM with a real scalar.




\subsection{Multi-field $Z_N$ dark matter}
If $N$ is prime there will be only a dark matter candidate.
If $N$ is not prime then the Lagrangian will preserve additional discrete symmetries that are subgroups of $Z_N$, allowing for the possibility of multiple DM candidates.

\subsubsection{$Z_4$ symmetry}




For $Z_4$ as remnant of the spontaneous breaking of some $\operatorname{U}(1)_X$,  we can have two-component dark matter depending on the mass hierarchy of the dark sector~\cite{Batell:2010bp,Aoki:2016glu}. For larger $Z_N$ see ~\cite{Batell:2010bp}. In~\cite{Chen:2018lsk} is claimed that the heterogeneous self-interaction is a natural consequence of any 2DM or nDM models. In nDM models there can be various dark matter (DM) annihilation processes that are different from the standard DM annihilation process~\cite{Aoki:2012ub}. In particular it is possible to have an assisted freze-out which further reduce the direct detection signals~\cite{Belanger:2011ww}

\bibliographystyle{apsrev4-1longdoi}
\bibliography{references}

\end{document}

