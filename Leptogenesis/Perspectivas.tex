La escala de leptogénesis juega un papel fundamental en el desarrollo de los modelos y en su consecuencia, de las posibles verificaciones experimentales. Existen diferentes modelos que se basan en esta tarea como lo son el re-escalamiento de $\nu$ y el modelo de doblete inerte, entre otros. Nosotros nos centraremos en el análisis de los modelos mas sobresalientes en estas áreas, además, prestaremos especial atención a la cosmología no estándar y su naciente relación con la leptogenesis. 
Dentro de la cosmología no estándar surge la idea de que la leptogenesis sucede a una escala mucho mas baja, de esta forma su confirmación experimental puede ser mas viable.