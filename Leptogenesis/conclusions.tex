%taking from \section*{adapted from the the SDFDM model with scalars conclusions}
El cumplimiento de las condiciones de A. Sakharov por parte de una teoría, han permitido delinear los posibles escenarios para la bariogénesis. En el caso de la leptogenesis, el decaimiento de neutrinos derechos fuera del equilibrio térmico genera una asimetría leptónica, la cual, es proporcional a una asimetría $CP$ denotada como $\epsilon$. De manera más precisa, $\epsilon$ denota la asimetría $CP$ que se origina en el decaimiento de los neutrinos.
%
% * <restrepo@udea.edu.co> 2018-05-29T15:06:56.592Z:
% 
% Conectar con generación de asimetrìa en número leptónico
% 
% ^.
%
Estableciendo $M_{R_{1}}\ll M_{R_{2}},M_{R_{3}}$ 
\begin{align}
|\epsilon|&\lesssim\frac{3}{16\pi}\frac{M_{1}}{v^2}(m_{3}-m_{1}).
\end{align}
 Por tanto, existe una dependencia entre $\epsilon$ y $M_{R_{1}}$. las restricciones que las asimetría realiza sobre $\epsilon$, obligan a establecer una cota sobre $M_{1}$, imponiéndose que $M_{1}$ deba ser mayor a $10^{9}$ $\textup{GeV}$. En los modelos de generación de asimetría bariónica, la leptogenesis es una de las que más destaca, sus bases están formadas en conceptos fundamentales de la física de neutrinos. La gran desventaja de la leptogenesis es que su demostración experimental esta aun lejos de ser posible
