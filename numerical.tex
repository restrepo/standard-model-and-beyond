
\chapter{Computational QFT}

\begin{frame}
There are several tools which allows for the generation of models of particle physics models like LanHEP \cite{Semenov:2008jy} 
\begin{center}
  \url{http://theory.sinp.msu.ru/~semenov/lanhep.html},
\end{center}
or FeynRules \cite{Christensen:2008py}
\begin{center}
  \url{http://feynrules.phys.ucl.ac.be/}\,.
\end{center}

This kind of programs are able to generate the output required for other programs which make the calculation of Feynman diagrams and integration over multi-particle phase space. CalcHEP:
\begin{center}
  \url{http://theory.sinp.msu.ru/~pukhov/calchep.html}
\end{center}
for example, is able to calculate cross section and decays widths at tree level.

\end{frame}

In this chapter we will illustrate the use LanHEP+CalcHEP

\section{LanHEP}

\begin{frame}[fragile,allowframebreaks]{}
After download the source code from \url{http://theory.sinp.msu.ru/~pukhov/calchep.html} to some \lstinline{DIR}, 
\begin{itemize}
\item Note that the \lstinline{tar.gz} file name depends on the current version. At the moment of this writing this was \lstinline{lhep311.tar.gz}. To uncompress the file:\\
  \lstinline{\$ tar -zxvf DIR/lhep311.tar.gz}\\
where \lstinline{\$} is to indicate that the command is to be written in the shell of your Linux session\footnote{An introduction to scientific computing is at \url{http://gfif.udea.edu.co/cf} }
\item Go to the created directory
  \lstinline{\$ cd lanhep311}

\item To compile and create the executable of the program, that will be called \lstinline{lhep}:\\
\lstinline{\$ make}  
\end{itemize}


\end{frame}

%%% Local Variables: 
%%% mode: latex
%%% TeX-master: "beyond.beamer.tex"
%%% End: 

